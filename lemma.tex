\begin{lemma}\label{lem:ref_combination}
For any $B \in \P' \in \Pi'$ and $\P_B \in \Pi'_B$, let $\P = \P_B \cup \P' \backslash \{B\} $
be the refinement of $\P'$. Then we have
\begin{equation}\label{eq:ref_combination}
I_{\P}(Z_V) = \theta I_{\P'}(Z_V) + (1-\theta) I_{\P_B}(Z_B)
\end{equation}
\end{lemma}
\begin{proof}
First we have $\abs{\P} = \abs{\P'} -1 + \abs{\P_B}$,
Since
\begin{align*}
I_{\P}(Z_V) & = { f[\P'] + f_B[\P_B] \over \abs{\P} - 1} \\
I_{\P'}(Z_V) & = { f[\P'] \over \abs{\P'} - 1} \\
I_{\P_B}(Z_B) & = { f_B[\P_B] \over \abs{\P_B} - 1}
\end{align*}
As a result, there exists $\theta = {\abs{\P'} - 1 \over \abs{\P} - 1} \in (0,1)$ such that 
\eqref{eq:ref_combination} holds.
\end{proof}

\begin{lemma}\label{lem:mi_split}
For any $\P:=\{C_1, \dots, C_k\} \in \Pi'$, we have
\begin{align}
I_{\P}(Z_V) =& I(Z_{C_1} \wedge \dots \wedge Z_{C_k}) \label{eq:mmi_representation} \\
=& { 1 \over k-1} \sum_{i=1}^{k-1} I(Z_{C_i} \wedge Z_{\cup_{j=i+1}^k C_j}) \label{eq:mi_mmi}
\end{align}
\end{lemma}
\begin{proof}
$I(Z_1 \wedge Z_2)$ is a kind of representation of mutual information.
And we use \eqref{eq:mmi_representation} to represent the multivariate mutual information.

Equation \eqref{eq:mi_mmi} is true for $k=2$. Suppose it holds for $k=m$. Then
let $B=\{C_m, C_{m+1}\}$ , $\P=\{C_1, \dots, C_{m+1}\}$
and use the result of lemma \ref{lem:ref_combination}, 
we have $\theta = \frac{m-1}{m}$
\begin{align*}
I(Z_{C_1} \wedge \dots \wedge Z_{C_{m+1}}) & = 
 \frac{m-1}{m} I(Z_{C_1} \wedge \dots \wedge Z_{C_m})
+ \frac{1}{m}  I(Z_{m} \wedge Z_{m+1}) \\
& = \frac{m-1}{m} \frac{1}{m-1}\sum_{i=1}^{m-1} I(Z_{C_i} \wedge Z_{\cup_{j=i+1}^m C_j})\\
& + \frac{1}{m}  I(Z_{m} \wedge Z_{m+1})\textrm{ by induction} \\
& = \frac{1}{m} \sum_{i=1}^{m} I(Z_{C_i} \wedge Z_{\cup_{j=i+1}^{m+1} C_j})
\end{align*}
\end{proof}
\begin{remark}
This lemma shows that we can use the mutual information to compute the multivariate mutual information.
\end{remark}
\begin{lemma}
If $I_{P}(Z_V) = I(Z_V)$, then for any $\P_B, B\in P$, we have $I_{P_B}(Z_B) \geq I(Z_V)$
\end{lemma}
\begin{proof}
By lemma \ref{lem:ref_combination}, let $\P' = \P_B \cup \P\backslash B$ we have 
\begin{equation}
I_{\P'}(Z_V) = \theta I_{\P}(Z_V) + (1-\theta) I_{\P_B}(Z_B)
\end{equation}
Since $I_{\P'}(Z_V)\geq I(Z_V)$, we have $I_{P_B}(Z_B) \geq I(Z_V)$.
\end{proof}
\begin{lemma}\label{lem:smallZB}
Suppose $I_{\P}(Z_V) = I(Z_V)$, then we have $I(Z_V) \geq I_{\P_B}(Z_B) \geq I(Z_B)$ if $\exists \P_B \subseteq \P$.
\end{lemma}
\begin{proof}
Let $\P' = \{B\} \cup \P\backslash \P_B$, then $\P = \P_B \cup \P'\backslash\{B\}$ , use lemma , we have
\begin{equation}
I_{\P}(Z_V) = \theta I_{\P'}(Z_V) + (1-\theta) I_{\P_B}(Z_B)
\end{equation}
Since  $I_{P}(Z_V) = I(Z_V), I_{\P'}(Z_V)\geq I(Z_V) \Rightarrow I(Z_V) \geq I_{\P_B}(Z_B) \geq I(Z_B)$ 
\end{proof}
\begin{lemma}
\begin{equation}\label{eq:P12Inequality}
\abs{\P_1} + \abs{\P_2} \leq \abs{\P_1 \cap \P_2} + \abs{\P_1 \cup \P_2}
\end{equation}
\end{lemma}
\begin{proof}
Let $\P_2 = \{B_1, \dots, B_r\} \cup \{\{i\}, i\not\in B_i\}$, where $\abs{B_i}>1$.
If $r=1$, \eqref{eq:P12Inequality} holds.
$\abs{\P_2} = 1+ \abs{V} - \abs{B}, \abs{\P_1 \vee \P_2} = \abs{\P_1} - \abs{\{C\in \P_1 | C\cap B \neq \emptyset \}}+1, \abs{\P_1 \wedge \P_2} = \abs{V} - \abs{B} + \abs{\{C\in \P_1 | C\cap B \neq \emptyset \}}$,
 therefore, $\abs{\P_1} + \abs{\P_2} = \abs{\P_1 \wedge \P_2} + \abs{\P_1 \vee \P_2}$
\end{proof}