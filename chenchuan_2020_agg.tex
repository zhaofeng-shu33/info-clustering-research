\documentclass{ctexart}
\usepackage{amsmath}
\begin{document}
    Chung Chan further summarized his work on Agglomerative Info-Clustering
    in a journal \cite{chan2020agglomerative}.
    In the actual implementation of Algorithm 2,
    a slight different notation is used. 
    That is: $j$ iterates from 2 to $k$,
    \begin{equation}
        g_j(B) = h(\bigcup_{i \in B \cup \{j\}} C_i)
        - \sum_{i \in B \cup \{j\} } h(C_i), \textrm{ for } B \subseteq \{1, \dots, j-1\}
    \end{equation}
    This can be treated as reversing the order of appearance of elements
    in $\mathcal{P}$ as 
    $\mathcal{P} = \{C_k, \dots, C_1\}$.

    Actually the implementation is not faster than the baseline implementation (See \cite{mac})
    to compute the
    principal sequence of partition. Though the theoretic
    contributions are thorough, I don't think this article contain
    much nutritional contents for practitioners of graph algorithms.
    \bibliographystyle{plain}
    \bibliography{exportlist}
\end{document}

