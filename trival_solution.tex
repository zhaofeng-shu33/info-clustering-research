\documentclass{article}
\usepackage{amsfonts}
\usepackage{amsmath}
\usepackage{amsthm}
\usepackage{algorithm}
\usepackage{algorithmic}
\usepackage{bm}
\usepackage{hyperref}
\usepackage{footmisc}
\usepackage{xcolor}
\DeclareMathOperator\E{\mathbb{E}}
\DeclareMathOperator\Var{\mathrm{Var}}
\def\R{\mathbb{R}}
\def\P{\mathcal{P}}
\usepackage{mathtools}
\DeclarePairedDelimiter\abs{\lvert}{\rvert}
\DeclarePairedDelimiter\norm{\lVert}{\rVert}
\DeclarePairedDelimiter\inner{\langle}{\rangle}
\def\red#1{\textcolor{red}{#1}}
\makeatletter
\newcommand{\algorithmicfunction}{\textbf{function}}
\newcommand{\algorithmicendfunction}{\algorithmicend\ \algorithmicfunction}
\newenvironment{ALC@func}{\begin{ALC@g}}{\end{ALC@g}}
\newcommand{\FUNCTION}[2][default]{\ALC@it\algorithmicfunction\ #2\ %
\textbf{:}%
\ALC@com{#1}\begin{ALC@func}}
\ifthenelse{\boolean{ALC@noend}}{
    \newcommand{\ENDFUNCTION}{\end{ALC@func}}
  }{
    \newcommand{\ENDFUNCTION}{\end{ALC@func}\ALC@it\algorithmicendfunction}
  }
\makeatother
\theoremstyle{definition}
\newtheorem{definition}{Definition}
\newtheorem{theorem}{Theorem}
\newtheorem{example}{Example}
\newtheorem{corollary}{Corollary}
\newtheorem{lemma}{Lemma}
\newtheorem{remark}{Remark}
\begin{document}
If the graph weights are similar, we cannot distinguish each cluster. 
In such case, info-clustering algorithm will provide trivial solution: 
that is, either the whole set is one cluster or each singleton forms one cluster.

We consider one extreme case where the graph is complete graph and each edge is with equal weight value (suppose equal to 1 for brevity).
Then we will show that the info-clustering solution is trivial.

\begin{lemma}
\begin{equation}\label{eq:larger_n_2}
\frac{f[\P]}{\abs{\P}-1} \geq \frac{n}{2}, \quad\forall \P
\end{equation}
where $n = \abs{V}$ and the equality holds only when $\abs{\P}=n$.
\end{lemma}
\begin{proof}
Math deduction on $\abs{\P}$ to prove $f[\P] \geq \frac{n}{2}(\abs{\P} - 1)$.  We know that the equality holds for $\abs{\P} = n$.
Suppose \eqref{eq:larger_n_2} holds for $\abs{\P} =k+1$. For $\abs{\P}=k(k\geq 2)$, the new paritition $\widetilde{\P}$ can be divided into $n_1\leq n_2 \dots n_k$ parts where $n_1 \leq \frac{n}{k}$.

Let the smallest part be further partitioned into two parts with $1$ and $n_1-1$ elements respectively.
Then using assumption that that the conclusion holds for $\abs{\P}=k+1$ we have
$$
n-1 + (n_1-1)(n-n_1+1) + \sum_{i=2}^k n_i(n-n_i) \geq nk
$$
That is $\sum_{i=1}^k n_i(n-n_i) \geq nk - 2(n_1-1)$
Since $2(n_1-1)<2n_1\leq 2\frac{n}{k} \leq n$ we have
$$
 \sum_{i=1}^k n_i(n-n_i)  > n(k-1)
$$
That it, strict inequality holds for $\abs{\P}=k$.
\end{proof}
\end{document}