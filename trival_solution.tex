\documentclass{article}
\usepackage{amsfonts}
\usepackage{amsmath}
\usepackage{amsthm}
\usepackage{algorithm}
\usepackage{algorithmic}
\usepackage{bm}
\usepackage{footmisc}
\usepackage{xcolor}
\DeclareMathOperator\E{\mathbb{E}}
\DeclareMathOperator\Var{\mathrm{Var}}
\def\R{\mathbb{R}}
\def\P{\mathcal{P}}
\usepackage{mathtools}
\DeclarePairedDelimiter\abs{\lvert}{\rvert}
\DeclarePairedDelimiter\norm{\lVert}{\rVert}
\DeclarePairedDelimiter\inner{\langle}{\rangle}
\def\red#1{\textcolor{red}{#1}}
\makeatletter
\newcommand{\algorithmicfunction}{\textbf{function}}
\newcommand{\algorithmicendfunction}{\algorithmicend\ \algorithmicfunction}
\newenvironment{ALC@func}{\begin{ALC@g}}{\end{ALC@g}}
\newcommand{\FUNCTION}[2][default]{\ALC@it\algorithmicfunction\ #2\ %
\textbf{:}%
\ALC@com{#1}\begin{ALC@func}}
\ifthenelse{\boolean{ALC@noend}}{
    \newcommand{\ENDFUNCTION}{\end{ALC@func}}
  }{
    \newcommand{\ENDFUNCTION}{\end{ALC@func}\ALC@it\algorithmicendfunction}
  }
\makeatother
\theoremstyle{definition}
\newtheorem{definition}{Definition}
\newtheorem{theorem}{Theorem}
\newtheorem{example}{Example}
\newtheorem{corollary}{Corollary}
\newtheorem{lemma}{Lemma}
\newtheorem{remark}{Remark}
\begin{document}
If the graph weights are similar, we cannot distinguish each cluster. 
In such case, info-clustering algorithm will provide trivial solution: 
that is, either the whole set is one cluster or each singleton forms one cluster.

We consider one extreme case where the graph is complete graph and each edge is with equal weight value (suppose equal to 1 for brevity).
Then we will show that the info-clustering solution is trivial.

\begin{lemma}
\begin{equation}\label{eq:larger_n_2}
\frac{f[\P]}{\abs{\P}-1} \geq \frac{n}{2}, \quad\forall \P
\end{equation}
where $n = \abs{V}$.
\end{lemma}
\begin{proof}
Math deduction on $n$ to prove $f[\P] \geq \frac{n}{2}(\abs{\P} - 1)$.  Suppose \eqref{eq:larger_n_2} holds for $n=k$ and every partition $\P$. For $n=k+1$, the new paritition $\widetilde{\P}$ has two choices: 
\begin{enumerate}
\item the $(k+1)$-th node forms one cluster itself, and $\tilde{\P} = \P \cup \{\{k+1\}\}$, 
then $f[\widetilde{\P}] = f[\P] + \abs{\P} \geq \frac{n}{2}(\abs{\P}-1) + \abs{\P} = \frac{n+1}{2}\abs{\P} $
\item the $(k+1)$-th node joined one of the previous cluster, denoted as $C_1$. Since $\abs{C_1} \leq \abs{n}-\abs{\P} + 1$, then 
$f[\widetilde{\P}] = f[\P]+ (n-\abs{C_1}) \geq \frac{n}{2}(\abs{\P}-1) + (n-\abs{C_1}) \geq \frac{n}{2}(\abs{\P}-1) +(\abs{\P} - 1) \geq \frac{n+1}{2}(\abs{\P}-1)$. That is $f[\widetilde{\P}]  \geq \frac{n+1}{2}(\abs{\widetilde{\P}}-1)$
\end{enumerate}
\end{proof}
\end{document}